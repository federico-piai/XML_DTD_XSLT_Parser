\addto\captionsfrench{\renewcommand*{\glossaryname}{Glossaire}}
\addto\captionsfrench{\renewcommand*{\acronymname}{Liste des accronymes}}

\newcommand{\gBox}[1]{% 
	\shadowbox{\sffamily#1}\\\hfill\mbox{}
}

\newglossarystyle{nation}{%
	% put the glossary in the itemize environment:
	\renewenvironment{theglossary}{\begin{description}}{\end{description}}%
	% have nothing after \begin{theglossary}:
	\renewcommand*{\glossaryheader}{}%
	% have nothing between glossary groups:
	\renewcommand*{\glsgroupheading}[1]{}%
	\renewcommand*{\glsgroupskip}{}%
	
	% set how each entry should appear:
	\renewcommand*{\glossaryentryfield}[5]{%
	\item % bullet point
	\gBox{\glstarget{##1}{\uppercase{##2}}}% the entry name
	%\space (##4)% the symbol in brackets
	\space {##3}% the description
	\space [Page(s) ##5]\\% the number list in square brackets
	}%
	% set how sub-entries appear:
	\renewcommand*{\glossarysubentryfield}[6]{%
		\glossaryentryfield{##2}{##3}{##4}{##5}{##6}}%
}
\glossarystyle{nation}


